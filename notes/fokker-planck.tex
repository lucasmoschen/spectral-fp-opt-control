\documentclass[12pt]{article}

\usepackage[english]{babel}
\usepackage[left=3cm, right=3cm, top=3cm, bottom=3cm]{geometry}
\usepackage{setspace}

\usepackage{amsmath, amssymb, amsfonts, amsthm}
\usepackage{cancel}
\usepackage{graphicx, float}
\usepackage{hyperref}
\usepackage{enumerate}

\title{Fokker-Plank equation}
\author{Lucas Moschen}
\date{Abril de 2021}

% mathematical definitions 
\newcommand{\R}{\mathbb{R}}
\newcommand{\N}{\mathcal{N}}
\newcommand{\A}{\mathcal{A}}
\newcommand{\B}{\mathcal{B}}
\renewcommand{\P}{\mathcal{P}}
\newcommand{\Q}{\mathcal{Q}}
\newcommand{\n}{\vec{n}}
\newcommand{\steady}{\rho_{\infty}}
\newcommand{\inner}[2]{\langle{} #1, #2 \rangle{}}

\newtheorem{theorem}{Theorem}[subsection]
\newtheorem{proposition}{Proposition}[subsection]
\newtheorem{remark}{Remark}[subsection]
\theoremstyle{definition}
\newtheorem{example}{Example}[subsection]
\newtheorem{definition}{Definition}[subsection]

% other
\newcommand{\sskip}{\vspace{5mm}}

\begin{document}

\maketitle

\onehalfspacing{}
    
\begin{abstract}
    This text aims to summarise the content studied during the period in the Department of Mathematics at Imperial College London with professor Dante Kalise.
    The object of study is the Fokker-Planck equation and Optimal Control.
\end{abstract}

\tableofcontents

\section*{Notation}

This list is part of the notation we use throughout the text:

\begin{itemize}
    \item $\langle X \rangle = \mathbb{E}[X]$ is the expected value of the random variable $X$.
    \item For $f, g \in H^1(\Omega)$, we define $\langle f, g \rangle = \int_{\Omega} f(x) g(x) \, dx$
\end{itemize}

\section{Introduction}

Consider the Stochastic Differential Equation
\[
dX_t = -\nabla V(x,t) \, dt + \sqrt{2\nu} dB_t.
\]
The probability density function $\rho$ of the process ${\{X_t\}}_{t \in \R_+}$ is the solution of 
\[
\rho_t(x,t) = \nabla \cdot J(x,t), \text{ where } J(x,t) = \nu \nabla \rho(x,t) + \rho(x,t) \nabla V(x,t), (x,t) \in \Omega \times (0,+\infty), 
\]
subject to $J(x,t) \cdot \n = 0$ on $\partial \Omega \times (0,+\infty)$ and $\rho(x,0) = \rho_0(x)$ in $\Omega$. 
Consider the potential 
\[
V(x,t) = G(x) + u(t) \alpha(x),
\]
such that $u$ is a control function and $\alpha$ is a control shape function that satisfies $\nabla \alpha \cdot \n = 0$ on $\partial \Omega$.
With that, we get that the boundary condition reduces to
\[
0 = (\nu \nabla \rho + \rho \nabla V) \cdot \n = \nu (\nabla \rho) \cdot \n + \rho (\nabla G) \cdot \n + \cancel{u~\rho (\nabla \alpha) \cdot \n},
\]
witch implies that 
\[
(\nu \nabla \rho + \rho \nabla G) \cdot \n = 0 \text{ on } \partial \Omega.
\]

Let $\steady$ be the steady state for the uncontrolled system, i.e., the function that solves 
\[
\nabla \cdot (\nu \nabla \steady + \steady \nabla G) = 0 \text{ in } \Omega \times (0,\infty),
\]
subject to $(\nu \nabla \steady + \steady \nabla G) \cdot \n = 0$ on $\partial \Omega$.
Define $y = \rho - \steady$.
The problem turns to 
\[
\begin{split}
    y_t &= \nabla \cdot \left[\nu \nabla (y + \steady) + (y + \steady) \nabla V\right] \\
    &= \nabla \cdot \left(\nu \nabla y + y \nabla V + u\nabla \alpha \steady\right) + \cancel{\nabla \cdot (\nu \nabla \steady + \steady \nabla G)},
\end{split}
\]
in $\Omega \times (0,\infty)$ subject to 
\[
0 = (\nu \nabla (y + \steady) + (y + \steady)\nabla G) \cdot \n = (\nu \nabla y + y\nabla G) \cdot \n \text{ on } \partial \Omega 
\]
and $y(x,0) = \rho_0(x) - \steady(x)$ in $\Omega$.
We can write the above problem as a bilinear control system 
\[
\dot{y} = \mathcal{A} y + u\mathcal{N}y + \mathcal{B} u, \quad y(0) = y_0,
\]
where $\mathcal{B} = \mathcal{N} \steady$.

The first feedback strategy is based on the cost functional 
\[
\mathcal{J}(y, u) = \frac{1}{2}\int_0^{\infty} \langle y, \mathcal{M} y \rangle + |u|^2 \, dt,    
\]
and it is obtained solving the Riccatti Equation
\[
\mathcal{A}^*\Pi + \Pi\mathcal{A} - \Pi\mathcal{B}\mathcal{B}^*\Pi + \mathcal{M} = 0
\]
and $u = -\mathcal{B}^{*}\Pi y$.

The second strategy is based on the solution $\Gamma$ to a Lyapunov equation
\[
\mathcal{A}^* \Upsilon + \Upsilon \mathcal{A} + 2 \mu I = 0,   
\]
for a properly chosen parameter $\mu > 0$.

Some additional references:

\begin{itemize}
    \item~\cite{bris2008existence}: covers the existence and the uniqueness of a class of Fokker-Planck equations with coefficients in Sobolov spaces $W^{1,p}$.
    \item~\cite{huang2015steady}: existence of steady state of Fokker-Planck equation in $\Omega \subseteq \R^n$.
\end{itemize}

\section{Well-posedness}

We search for a solution $\rho \in W(0,T) = L^2(0,T; H^1(\Omega)) \cap H^1(0, t; {(H^1(\Omega))}^*)$ and 
\[
\inner{\rho_t}{\phi} + \inner{\nu\nabla \rho(t) + \rho(t)\nabla G}{\nabla \phi} + u(t)\inner{\rho(t)\nabla \alpha}{\nabla \phi} = 0, \forall \phi \in H^1(\Omega),
\]
where integration by parts was applied.
It is assumed that 
\[
G, \alpha \in W^{1,\infty}(\Omega) \cap W^{2,\max(2,n)}(\Omega).
\]
With that in mind, is possible to prove that for every $u \in L^2(0,T)$ and $\rho_0 \in L^2(\Omega)$, there exists an unique solution to Fokker-Planck equation.

Moreover the solution satisfies
\begin{enumerate}[(i)]
    \item For every $t \in [0,T]$, we have that $\inner{\rho(t) - \rho_0}{1_{\Omega}} = 0$.
    \item If $\rho_0 \ge 0$ almost everywhere on $\Omega$, then $\rho(x,t) \ge 0$ for all $t > 0$ and almost all $x \in \Omega$.
\end{enumerate}

\section{Operator form}

Define the operators 
\[
\begin{split}
    \mathcal{A} &: \mathcal{D}(\mathcal{A}) \to L^2(\Omega), \\
    &\mathcal{D}(\mathcal{A}) = \{\rho \in H^2(\Omega) : (\nu \nabla \rho + \rho \nabla G) \cdot \n = 0 \text{ on } \partial \Omega\} \\
    &\mathcal{A} \rho = \nu \Delta \rho + \nabla \cdot (\rho \nabla G),   
\end{split}
\]
and
\[
\mathcal{N} : H^1(\Omega) \to L^2(\Omega), \mathcal{N}\rho = \nabla \cdot (\rho \nabla \alpha).    
\]
The adjoint operators are well-defined and given by 
\[
\mathcal{A}^*\varphi = \nu \Delta \varphi - \nabla G \cdot \nabla \varphi, (\nu \nabla \varphi) \cdot \n = 0 \text{ on } \partial \Omega    
\]
and 
\[
\mathcal{N}^* \varphi = - \nabla \varphi \cdot  \nabla \alpha. 
\]

Considering the uncontrolled system $\dot{\rho} = \mathcal{A} \rho$, introduce the function $\Phi(x) = \log \nu + \frac{G(x)}{\nu}$.
Further, define the operator $\mathcal{A}_s = e^{\Phi/2}\mathcal{A}e^{-\Phi/2}$.
Then,
\[
\mathcal{A}\left(e^{-\Phi/2} \rho \right) = \nu e^{-\Phi/2}\left(\Delta \rho + \frac{1}{2}\rho\Delta \Phi - \frac{1}{4}\rho \nabla \Phi \cdot \nabla \Phi \right).
\]
The following results can be obtained:
\begin{itemize}
    \item $\mathcal{A}_s$ is self-adjoint.
    \item The spectrum $\sigma(\mathcal{A}_s)$ consists of non-positive pure points with $0 \in \sigma(\mathcal{A}_s)$.
    \item The eigenfunctions ${\{\psi_i\}}_{i=0}^{\infty}$ form a complete orthogonal set.
    \item $\sigma(\mathcal{A}_s) = \sigma(\mathcal{A})$ and $\psi_i$ is eigenfunction of $\mathcal{A}$ iff $e^{\Psi/2}\psi_i$ is eigenfunction of $\mathcal{A}_s$. 
    \item $\psi_i$ is eigenfunction of $\mathcal{A}$ iff $e^{\Psi}\psi_i$ is eigenfunction of $\mathcal{A}^*$. 
    \item $\steady = ce^{-\Phi}$ is eigenfunction of $\mathcal{A}$ associated with the eigenvalue $0$.
\end{itemize}

\subsection{Decoupling the Fokker-Planck equation}

Consider the projection transform $\mathcal{P}$ onto $1^{\perp}$ along $\steady$, written as 
\[
\mathcal{P}y = y - \int_{\Omega} y \, dx \steady,    
\]
and its complementary $\mathcal{Q}y = (I-\mathcal{P})y$.
If $y \in \mathcal{Y}$, we can decompose $\mathcal{Y} = im(\mathcal{P}) \oplus im(\mathcal{Q})$, which implies $y = \mathcal{P}y + \mathcal{Q}y = y_{\mathcal{P}} + y_{\mathcal{Q}}$.
Therefore, Fokker-Planck equation turns to
\[
\dot{y}_{\P} + \dot{y}_{\Q} = \mathcal{A}(y_{\P} + y_{\Q}) + u\N(y_{\P} + y_{\Q}) + \B u.
\]
Applying $\P$ and $\Q$ to this equation and noticing that $\P(\P y) = \P y, \Q(\Q y) = \Q y, \P(\Q y) = \Q(\P y) = 0$, we get the system
\[
\begin{bmatrix}
    \dot{y}_{\P} \\ \dot{y}_{\Q} 
\end{bmatrix} = \begin{bmatrix}
    \P\A & 0 \\ 0 & 0 
\end{bmatrix}\begin{bmatrix}
    y_{\P} \\ y_{\Q} 
\end{bmatrix} + u\begin{bmatrix}
    \P\N & \P\N \\ 0 & 0 
\end{bmatrix}\begin{bmatrix}
    y_{\P} \\ y_{\Q} 
\end{bmatrix} + u\begin{bmatrix}
    \P\B \\ 0
\end{bmatrix},
\]
where other identities were also used.

\section{Ricatti-based feedback control}

It is considered the linearised version of the system, that is, $uNy$ term is dropped.
In this section, they derive the Ricatti equation.

\section{Methods for solving a PDE}

\begin{enumerate}[(a)]
    \item Finite differences:~\cite{chang1970practical} build a scheme where positivity and conservation of mass are maintained.
    \item Finite elements
    \item Collocation methods, which is a spectral method in the strong sense
    \item Spectral-Legendre method
\end{enumerate}

The equation we are trying to solve is 
\[
y_t = \nabla \cdot (\nu \nabla y + y \nabla G + u y \nabla \alpha + u \steady \nabla \alpha),
\]
subject to $(\nu \nabla y + y \nabla G) \cdot \n = 0$ in the boundary and $y = \rho_0 - \steady$ as initial condition.
In the weak formulation, for every $\phi \in H^1(\Omega)$, we have
\[
\inner{y_t}{\phi} = -\inner{\nu \nabla y + y \nabla G}{\nabla \phi} - u(t)\inner{\nabla \alpha y}{\nabla \phi} - u(t)\inner{ \nabla \alpha \steady}{\nabla \phi}.
\]

\subsection{Spectral-Legendre method}

Let us consider the one-dimensional case for now: $\Omega = [a,b]$.
To simplify future calculations, we consider the variable
\[
\tilde{y}(x,t) = y\left(\left(\frac{b-a}{2}\right)x + \left(\frac{a+b}{2}\right), t\right), \forall x \in [-1,1], t > 0.    
\]
We do the same for $G$, $\alpha$ and $\steady$. 
For sake of conciseness, we drop $\sim$ for now.
The formulation turns to 
\[
\inner{y_t}{\phi} = - {\left(\frac{2}{b-a}\right)}^2\inner{\nu y_x + y \dot{G} + u y \dot{\alpha} + u\steady \dot{\alpha}}{\dot{\phi}}.
\]

Consider the space 
\[
X_n = \{\phi \in P_n : \nu \dot{\phi}(\pm 1) + \phi(\pm 1) \dot{G}(\pm 1) = 0\},
\]
where $P_n$ is the space of polynomials with degree up to $n$.
Notice that $\dim(X_n) = n-1$.
Let ${\{\phi_i\}}_{i=0}^{n-2}$ be a basis for $X_n$ and write 
\[
y(x,t) \approx \sum_{j=0}^{n-2} y_j(t) \phi_j(x). 
\]
Considering the set of test equal to the trial functions, we get in the following formulation
\[
\begin{split}
    \sum_{j=0}^{n-2} \dot{y}_j(t) \inner{\phi_j}{\phi_i} = -{\left(\frac{2}{b-a}\right)}^2 &\sum_{j=0}^{n-2} y_j(t) \left(\nu \inner{\dot{\phi}_j}{\dot{\phi}_i} + \inner{\dot{G} \phi_j}{\dot{\phi}_i} + u(t)\inner{\dot{\alpha}\phi_j}{\dot{\phi}_i} \right) \\ 
    &+ u(t) \inner{\dot{\alpha} \steady}{\dot{\phi}_i},
\end{split}
\]
which can be rewritten as a system of ODEs:
\[
\Phi \dot{y}(t) = -{\left(\frac{2}{b-a}\right)}^2 (\Lambda + \Theta^1 + u(t)\Theta^2)y(t) -\frac{2}{b-a}u(t)v.    
\]
Following the suggestion from $\cite[p.7]{shen2011spectral}$, we consider
\[
\phi_k(x) = L_k(x) + \alpha_k L_{k+1}(x) + \beta_k L_{k+2}(x),    
\]
where $L_k$ is the Legendre polynomial of degree $k$ and the coefficients chosen as to satisfy the boundary conditions. 
Let us calculate this quantities:

\scalebox{0.85}{$
\nu (\dot{L}_k(\pm 1) + \alpha_k \dot{L}_{k+1}(\pm 1) + \beta_k \dot{L}_{k+2}(\pm 1)) + \dot{G}(\pm 1)(L_k(\pm 1) + \alpha_k L_{k+1}(\pm 1) + \beta_k L_{k+2}(\pm 1)) = 0.
$}

which can be written in matrix formulation
\[
\begin{bmatrix}
    \nu (k+1)(k+2) - 2g_{-} & -\nu (k+2)(k+3) + 2g_- \\
    \nu (k+1)(k+2) + 2g_+ & \nu (k+2)(k+3) + 2g_+
\end{bmatrix}\begin{bmatrix}
    \alpha_k \\ \beta_k
\end{bmatrix} = \begin{bmatrix}
    \nu k(k+1) - 2g_- \\ -\nu k(k+1) - 2g_+
\end{bmatrix},
\]
where $g_{\pm} = \dot{G}(\pm 1)$ and the system is numerically solved.

Let's now pre-calculate the matrices in terms of the legendre polynomials.

\[
\begin{split}
    \Phi_{ij} = \inner{\phi_i}{\phi_j} &= \inner{L_i}{L_j} + \alpha_i\inner{L_{i+1}}{L_j} + \beta_i\inner{L_{i+2}}{L_j} \\
    &+ \alpha_j\inner{L_i}{L_{j+1}} + \alpha_i\alpha_j\inner{L_{i+1}}{L_{j+1}} + \beta_i\alpha_j\inner{L_{i+2}}{L_{j+1}} \\
    &+ \beta_j\inner{L_i}{L_{j+2}} + \alpha_i\beta_j\inner{L_{i+1}}{L_{j+2}} + \beta_i\beta_j\inner{L_{i+2}}{L_{j+2}} \\
    &= 2\frac{\delta_{ij}}{2i + 1} + 2\alpha_i\frac{\delta_{i+1,j}}{2(i+1) + 1} + \beta_i\frac{\delta_{i+2,j}}{2(i+2) + 1} \\
    &= 2\alpha_j\frac{\delta_{i,j+1}}{2i + 1} + 2\alpha_i\alpha_j\frac{\delta_{ij}}{2(i+1) + 1} + 2\beta_i\alpha_j\frac{\delta_{i+1,j}}{2(i+2) + 1} \\
    &= 2\beta_j\frac{\delta_{i,j+2}}{2i + 1} + 2\alpha_i\beta_j\frac{\delta_{i,j+1}}{2(i+1) + 1} + 2\beta_i\beta_j\frac{\delta_{ij}}{2(i+2) + 1}
\end{split}
\]
\[
\begin{split}
    \Lambda_{ij} = \nu\inner{\dot\phi_i}{\dot\phi_j} &= \nu[\inner{\dot L_i}{\dot L_j} + \alpha_i\inner{\dot L_{i+1}}{\dot L_j} + \beta_i\inner{\dot L_{i+2}}{\dot L_j} \\
    &+ \alpha_j\inner{\dot L_i}{\dot L_{j+1}} + \alpha_i\alpha_j\inner{\dot L_{i+1}}{\dot L_{j+1}} + \beta_i\alpha_j\inner{\dot L_{i+2}}{\dot L_{j+1}} \\
    &+ \beta_j\inner{\dot L_i}{\dot L_{j+2}} + \alpha_i\beta_j\inner{\dot L_{i+1}}{\dot L_{j+2}} + \beta_i\beta_j\inner{\dot L_{i+2}}{\dot L_{j+2}}],
\end{split}
\]
where $\inner{\dot L_i}{\dot L_j} = L_{\max(i,j)}(1) \dot L_{\min(i,j)}(1) - L_{\max(i,j)}(-1) \dot L_{\min(i,j)}(-1)$, integrating by parts and observing that Legendre polynomials are orthogonal to other polynomials with smaller degree.
Therefore, 
\[
\inner{\dot L_i}{\dot L_j} = \frac{1}{2} \min(i,j) (\min(i,j) + 1)\left(1 + {(-1)}^{i+j}\right).
\]
Finally, we calculate
\[
\Theta^1_{ij} = \inner{\dot{G} \phi_j}{\dot \phi_i}, \Theta^2_{ij} = \inner{\dot{\alpha} \phi_j}{\dot \phi_i} \text{ and } v_i = \inner{\dot\alpha \steady}{\dot\phi_i}    
\]

This is the method of solving the PDE.\@
Before solving it, we need to compute the optimal control.
For that, we have to discretise the operators $\A, \B$ and $\mathcal{M}$:

The discretised versions are 
\[
A_{ij} = \inner{\A \phi_j}{\phi_i} = -\inner{\nu \dot \phi_j + \dot G \phi_j}{\dot \phi_i} = -\Lambda_{ij} - \Theta^1_{ij} \implies A = -(\Lambda + \Theta^1) 
\]
and 
\[
B_{i} = \inner{\B\cdot 1}{\phi_i} = -v_i \implies B = -v.
\]
With that in mind, we have to solve the discrete Ricatti equation
\[
A^T\Pi + \Pi A + \Pi B B^T \Pi + M = 0.
\]
with $u(t) = -B^T\Pi y(t)$. 
With this feedback, we solve
\[
\begin{split}
    \Phi \dot{y} &= {\left(\frac{2}{b-a}\right)}^2(A + \Theta^2B^T\Pi y(t) )y(t) - {\left(\frac{2}{b-a}\right)}^2BB^T\Pi y(t) \\ 
    &= {\left(\frac{2}{b-a}\right)}^2(A - BB^T\Pi  + \Theta^2B^T\Pi y(t))y(t).
\end{split}
\]
After solving this system, we have to come back to the original coordinate system.

\begin{remark}
    Notice that $M_{ij} = \inner{\mathcal{M} \phi_j}{\phi_i}$.
    For instance, if $\mathcal{M}$ is the identity operator, we have $M = \Lambda$.
\end{remark}

%%%%%%%%%%%%%%%%%%%%%%%%%%%%%%%%%%%%%%%%%%%%%%%
%                 Bibliografia                %
%%%%%%%%%%%%%%%%%%%%%%%%%%%%%%%%%%%%%%%%%%%%%%%

\bibliographystyle{plain}
\bibliography{biblio.bib}

\end{document}