\documentclass[12pt]{article}

\usepackage[english]{babel}
\usepackage[left=3cm, right=3cm, top=3cm, bottom=3cm]{geometry}
\usepackage{setspace}

\usepackage{amsmath, amssymb, amsfonts, amsthm}
\usepackage{graphicx, float}
\usepackage{hyperref}

\title{Fokker-Plank equation}
\author{Lucas Moschen}
\date{Abril de 2021}

% mathematical definitions 
\newcommand{\R}{\mathbb{R}}
\newcommand{\N}{\mathbb{N}}

\newtheorem{theorem}{Theorem}[subsection]
\newtheorem{proposition}{Proposition}[subsection]
\newtheorem{remark}{Remark}[subsection]
\theoremstyle{definition}
\newtheorem{example}{Example}[subsection]
\newtheorem{definition}{Definition}[subsection]

% other
\newcommand{\sskip}{\vspace{5mm}}

\begin{document}

\maketitle

\onehalfspacing{}
    
\begin{abstract}
    This text aims to summarise the content studied during the period in the Department of Mathematics at Imperial College London with professor Dante Kalise.
    The object of study is the Fokker-Planck equation and Optimal Control.
\end{abstract}

\tableofcontents

\section*{Notation}

This list is part of the notation we use throughout the text:

\begin{itemize}
    \item $\langle X \rangle = \mathbb{E}[X]$ is the expected value of the random variable $X$.
\end{itemize}

\section{Stochastic Differential Equations}
The most used Langevin equation is of the form
\[
\frac{dx}{dt} = a(x,t) + b(x,t) \xi(t),    
\]
where $\xi(t)$ is a random term.
We suppose that for $t \neq t'$, $\xi(t)$ and $\xi(t')$ are independent.
It is also assumed that $\langle \xi(t) \rangle = 0$.
This is not restrictive since $a(x,t)$ can incorporate the mean of $\xi(t)$.
Finally, we require that $\xi(t)$ has infinity variance.
The problem with this choice is the discontinuities of $dx/dt$.
Let 
\[
u(t) = \int_0^t  \xi(s) \, ds 
\]
and suppose that $u$ is a continuous function. 
This implies that $u$ is a Markov Process.
Notice that 
\[
\langle u(t + \Delta t) - u_0 | [u_0, t] \rangle = \langle \int_t^{t + \Delta t} \xi(s) \, ds \rangle = 0
\]
and 
\[
\langle {[u(t + \Delta t) - u_0]}^2 | [u_0, t] \rangle =  
\]

TOFINISH

\subsection{Stochastic Integration}

Suppose $G$ is an arbitrary function and $W(t)$ is a Wiener process.
The stochastic integral
\[
I := \int_{t_0}^t G(s) dW(s)    
\] 
is a kind of Riemann-Stieltjes integral. 
More specifically, consider the sum
\[
S_n = \sum_{i=1}^n G(\tau_i)[W(t_i) - W(t{i-1})], 
\]
with $t_0 \le t_1 \le \dots \le t_{n-1} \le t$ and $\tau_i \in [t_{i-1}, t_i]$,
The convergence of $S_n$, as this, depends on the choice of the values of $\tau_i$.
By this reason, we specify that $\tau_i = t_{i-1}$ leading to the {\em Itô Stochastic Integral}.

\begin{remark}
    The limit taken for $S_n$ is the mean square limit, that is, we want that 
    \[
    \lim_{n \to \infty} \langle {(S_n - I)}^2 \rangle = 0.
    \]
\end{remark}

An alternative definition comes from {\em Stratonovich}.


%%%%%%%%%%%%%%%%%%%%%%%%%%%%%%%%%%%%%%%%%%%%%%%
%                 Bibliografia                %
%%%%%%%%%%%%%%%%%%%%%%%%%%%%%%%%%%%%%%%%%%%%%%%

\begin{thebibliography}{9}
    \bibitem{main-book}

\end{thebibliography}


\end{document}